%% start of file `cv.tex'.
%% Copyright 2006-2015 Xavier Danaux (xdanaux@gmail.com).
%
% This work may be distributed and/or modified under the
% conditions of the LaTeX Project Public License version 1.3c,
% available at http://www.latex-project.org/lppl/.


\documentclass[11pt,a4paper,sans]{moderncv}

% moderncv themes
\moderncvstyle{casual}
\moderncvcolor{orange}

% character encoding
\usepackage[utf8]{inputenc}
\usepackage[T1]{fontenc}
\usepackage[icelandic,english]{babel}

% adjust the page margins
\usepackage[scale=0.75]{geometry}

% personal data
\name{Bjarni Jens}{Kristinsson}
\title{Curriculum Vitae}
\address{Hammarby allé 9}{12032 Stockholm}{Sweden}
\phone[mobile]{+46~(0)~73-515~00~26}
\email{bjarni.jens@gmail.com}
\homepage{bjk.is}
\social[linkedin]{bjk17}
\social[github]{bjk17}
\extrainfo{Last updated on \today}
\photo[64pt][0.4pt]{me.jpg}
%~ \quote{"It is quality rather than quantity that matters."\\
%~  \emph{-Lucius Annaeus Seneca,} Moral Letters to Lucilius}


\begin{document}
\makecvtitle

\section{Work experience}
\cventry{2019 -- \emph{curr.}}{Senior DevOps Consultant}{Opsdis}{Stockholm}{Sweden}
{Helping clients get a good overview of their operations using well-known
monitoring tools and data analysis techniques.}
\cventry{2015 -- 2017\\2018}{Software Developer}{WuXi NextCODE}{Reykjavik}{Iceland}
{Started off in DevOps like assignments of maintaining, executing and
further developing deployment (Chef, Ansible) and infrastructure (AWS
CloudFormation, Terraform) code. Handed it over to a newly created Backend
Group and joined the Data Group. Developed Python code to import and process
genomic data in our system and integrating 3rd party platforms with ours.
Built, tested and deployed components in CI/CD loops using tools such as
Jenkins, Docker and Ansible. Spent the summer of 2018 creating a benchmarking suite
for the company's core software which runs in a CI loop to detect performance
regression.}
%~ \cventry{Summer 2015}{Software Developer}{Handpoint}{Kopavogur}{Iceland}
%~ {Implementing protocols with partners and developing card reader software.
%~ Didn't like it and quit.}
%~ \cventry{Summer 2014}{Software Developer}{Invector}{Reykjavik}{Iceland}
%~ {Developing a web app for clients using Invector's statistical model to estimate
%~ prices of real estates worldwide. Working primarily on designing the database,
%~ the user system and other backend programming.}
%~ \cventry{Summer 2013}{Software Developer}{Reykjavik Energy}{Reykjavik}{Iceland}
%~ {Brought in to program an interactive educational game about Reykjavik Energy's
%~ CarbFix project. Worked with three Master's students who designed the game and
%~ wrote the educational material. Coded the game in raw JavaScript using images
%~ and graphics drawn and provided by them.}
%~ \cventry{Summer 2013}{Web programming}{Reconesse}{Reykjavik}{}
%~ {Together with two other university students we developed an
%~ interactive educational game about interesting female role models in
%~ women's right history for their website. The project received a
%~ grant from Rannís' Icelandic Student Innovation Fund.}

%~ \pagebreak
\section{Development tools}
\cvitem{Languages}{Python, Java}
\cvitem{Mindset}{DevOps, TDD, CI/CD, automation, pipelines, observability, end-to-end ownership}
\cvitem{Observability}{NumPy, R, Prometheus, InfluxDB, Grafana}
\cvitem{Toolbox}{Linux, Docker, Ansible, LXD, Jenkins, Travis, Vagrant, Bash/Zsh, git}
\cvitem{Spare parts}{Keras, TensorFlow, Octave/MatLab, SQL, \LaTeX}

\section{Education}
\cventry{2017 -- 2019}{M.Sc.\@ in Computer Science}{Reykjavik University}{Iceland}
{\textit{9,12} (out of 10)}{Spent the year 2017-18 at \emph{Vrije Universiteit Amsterdam}
in the Netherlands taking courses on distributed systems, concurrency algorithms, coding
theory and cryptography. RU courses on topics such as machine learning, deep
neural networks and combinatorics. MSc thesis named \emph{Searching for
combinatorial covers using integer linear programming} (available from
\url{http://hdl.handle.net/1946/34919}).}
\cventry{2012 -- 2015}{B.Sc.\@ in Mathematics}{University of Iceland}{Reykjavik}
{\textit{8,55} (out of 10)} {Specialization in Computer Science. Elective courses
in subjects such as algorithms, probability theory, combinatorics and graph theory.
President of the student union Stigull during the school year 2013-14. Wrote a
thesis named \emph{Occurrence graphs of patterns in permutations} (available from
\url{http://hdl.handle.net/1946/22017}) that got published in Involve
(\url{https://doi.org/10.2140/involve.2019.12.901}).}
%~ \cventry{2007 -- 2011}{Stúdentspróf}{Reykjavik Junior College}{Reykjavik}
%~ {\textit{7,89} (out of 10)}{Physics department. Received an acknowledgement
%~ for excellent results in Mathematics at graduation. Voted class councillor
%~ in final year.}

%~ \section{Master thesis}
%~ \cvitem{title}{Searching for combinatorial covers using integer linear programming}
%~ \cvitem{supervisor}{Henning A. Ulfarsson, Assistant Professor at Reykjavik University \newline Christian Bean, Postdoctoral Researcher at Reykjavik University}
%~ \cvitem{abstract}{We introduce the \textsf{CombCov} framework which 
%~ is a generalization of the \textsf{Struct} algorithm introduced by 
%~ Bean, Gudmundsson, and Ulfarsson in 
%~ \emph{``Automatic discovery of structural rules of permutation classes''}. 
%~ We give a simple example of an application of the framework to 
%~ \emph{avoidance sets} of \emph{words} and discuss in detail how to 
%~ generate \emph{rules} of lesser complexity and how a cover is verified 
%~ up to a certain size using integer linear programming. We then apply the
%~ framework to various published results on \emph{permutations} avoiding 
%~ \emph{mesh patterns} and try to find covers of similar problems with 
%~ some success. We show that \textsf{CombCov} is a powerful tool in 
%~ guiding humans by coming up with conjectures that would otherwise 
%~ have required substantial effort to discover manually.}
%~ \cvitem{url}{\url{http://hdl.handle.net/1946/34919}}

%~ \pagebreak
%~ \section{Bachelor thesis}
%~ \cvitem{title}{Occurrence graphs of patterns in permutations}
%~ \cvitem{supervisor}{Henning A. Ulfarsson, Postdoctoral Researcher at Reykjavik University}
%~ \cvitem{abstract}{This paper is based on a generalization of the
%~ idea behind the proof of the \emph{Simultaneous Shading Lemma} by
%~ Claesson et al.\@ (2014). We define the \emph{occurrence graph}
%~ $G_p(\pi)$ of a pattern $p$ in a permutation $\pi$ as the graph with
%~ the occurrences of $p$ in $\pi$ as vertices and edges between the
%~ vertices if the occurrences differ by exactly one element. We study
%~ the general properties of the occurrence graphs and some interesting
%~ extreme cases. The main theorem in this paper is that every \emph
%~ {hereditary property} of graphs produces a \emph{permutation class}.}
%~ \cvitem{url}{\url{http://hdl.handle.net/1946/22017}}
%~ \cvitem{url}{\url{https://arxiv.org/abs/1607.03018} \emph{(preprint)}}
%~ \cvitem{url}{\url{https://doi.org/10.2140/involve.2019.12.901} \emph{(publication)}}

\section{Teaching}
\cventry{2018 -- 2019}{Teacher Assistant}{Reykjavik University}{Reykjavik}{}
{Teacher assistant and grading homework in courses on Calculus, Statistics
and Discrete Mathematics.}
\cventry{Fall 2013}{Teacher Assistant}{University of Iceland}{Reykjavik}{}
{Teacher assistant in a Linear Algebra course.}
\cventry{2012 -- 2014}{More teaching}{Various employers}{Reykjavik}{}
{I have taught a computer science class for the Youth University
(summer 2013), revision courses in mathematics for Nobel tutoring Ltd.
(2012 -- 2013) as well as I have had many students for private tutoring
in mathematics (2012 -- 2014).}

\section{Interests}
\cvitem{tech}{Recent hands-on hobby projects include autonomous
    Raspberry Pis hosting websites and recording timelapses. Set
    up \url{blog.bjk.is} to document some of it. Manage my own VPS
    hosting websites and experimenting with various webservices.}
\cvitem{math}{I am deeply intrigued by the concept of infinity
    and I take joy in intuitive proofs by contradiction. During
    junior college I participated in multiple mathematics competitions
    and twice I was selected to compete with the national team in
    Baltic Way.}
\cvitem{chess}{For many years I studied chess and I achieved a peak
    ELO rating of 2062. In 2007 I became national champion U20 and
    in 2009 I played in the World Youth Chess Championship held in
    Antalya, Turkey. Twice I became Nordic champion and three times
    national champion with my junior college chess team. I have
    taught chess in various elementary schools and organized my own
    summer chess workshops.}
\cvitem{sports}{I like to lift weights, play football and bike to
    keep me in shape and in good health. I also bike to commute. In
    the summer of 2012 my friend and I went on a 2 month biking tour
    through Europe, visiting six countries and bicycling over 1600 km.}

%~ \pagebreak
\section{Languages}
\cvitemwithcomment{Icelandic}{native language}{}
\cvitemwithcomment{Swedish}{fluent}{}
\cvitemwithcomment{English}{full professional proficiency}{}
\cvitemwithcomment{French}{beginner level}{}

\section{References}
\cvitem{}{Available upon request.}


\end{document}


%% end of file `cv.tex'.
